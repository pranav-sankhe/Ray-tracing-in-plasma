\documentclass{resoResearch}
\usepackage{multicol,tfrupee}  

\usepackage{hyperref}


%%This is Resonance Research news sample
%%download resoResearch.cls and logo.eps and save it in the folder of  your source file
%%
%%use LaTeX, dvips and ps2pdf to compile and create PDFs
%%
%%The following packages are included with the class file.
%%Please download if these packages are not included
%%in your local TeX distribution 
%%txfonts,amssymb,amsfonts,amsthm,graphicx,geometry and amsmath
%%


\begin{document}
%%include \monthyear{month year} for month and year of publication in the footer

\monthyear{January 2016}

%%use \artNature for the running head information
\artNature{RESEARCH NEWS}

\begin{multicols}{2}
%%paper title
%%For line breaks, \\ can be used within title 
\title{Venom Evolution}
%%\secondTitle is optional
\secondTitle{Genetic and External Factors}
\author{Ema Fatima}
\maketitle






 The term venom is used for a variety of toxins
that  are  injected  by  certain  animals  into  a
victim  through  a  specialized  apparatus.
Though venom is most commonly employed
as a means of defense and predation, it is also
used as a means of asserting dominance over
conspecifics.  Venomous animals  include sea
anemones, jellyfish, gastropods, cephalopods,
centipedes, insect groups, echinoderms, some
species  of  amphibians,  fish,  snakes,  reptiles
and five species of mammals including shrews
and platypus. Venom can be hemotoxic, neurotoxic and cytotoxic and produce symptoms
that extend across the respiratory, muscular,
renal and gastrointestinal systems.

Emily S W Wong and Katherine Belov studied  the  origin  and  diversification  of  venom
genes  through  gene  duplication  which  is
known to be an important evolutionary force.
Gene duplication is the process of duplicating
a region of DNA; it is believed that the new
copy of  the gene mutates without deleterious
consequences to the organism and potentially
increases the fitness of the organism. In 2009,
Fry
et al
, suggested that venom sequences are
similar  at  a  molecular  level.  The  complex
cocktail of  compounds that  make up  venom
are believed to have arisen through gene duplication. The duplication of genes is a source
of  raw  genetic  substrate  from  which  novel
functions  arise.   Once  duplicated,  some  are
retained by natural selection or  genetic drift
while the majority are lost due to the accumulation of deleterious mutations. Mutations are
allowed to accumulate due to the presence of
an additional gene copy which buffers against
potential  deleterious  effects,  allowing  mutations to accrue over time for adaptive changes
to occur. Immediately after a duplication event,
duplicated toxins increase venom dosage and
accelerate  venom  replenishment.  Duplicates
increase  the  functional  diversity  of  venom
through  drift  and  selection  allowing an  animal to target a diverse range of molecules in
many  species.  The  duplicated genes can  act
cooperatively to induce synergistic effects in
victims.

Although it is believed that gene duplication
followed by adaptive selection is the primary
driver of venom evolution, not much analysis
has  been  possible  due  to  the  lack  of  fully
sequenced  genomes  of  venomous  animals.
Therefore, the platypus sequence was studied
to quantify the role of gene duplication in the
evolution of venom.  16  out of 107 platypus
genes  with  known  toxins  evolved  through
gene duplication events.   The lack of extensive gene duplications in platypus venom is at
odds  with  the  large  venom  gene  expansion
reported  in  snails  and  spiders  implying  that
selective pressures vary based on venom function. Platypus venom is present only in males
and is used for asserting dominance over competitors during the breeding season.

On  the  other  hand,
in  species  such  as  cone
snails  and  snakes,  where  venom  has  have
evolved through extensive duplications, adaptive selection is likely to have played a much
greater  role,  producing  genes  with  a  wide
diversity of functions. Sites under selection in
toxins may be influenced by a combination of
factors, including an animal’s feeding habits,
their  environment  and  biogeographical  factors. Although elucidating the functional significance of these site-specific changes can be
challenging,  the identification  of  sites  under
positive selection represents the first  step to
fully characterizing the molecular  actions of
toxins and their evolution.

Certain toxin gene families are known to repeatedly  evolve  through  gene  duplications.
The  rapidly  duplicating  gene  family  which
has  a  higher  likelihood of  generating potentially adaptive changes are large in size and
are capable of adapting to a number  of biological functions.  The other  kind  belongs  to
gene  families  associated  with  immunity  and
defense which are co-opted to toxin roles. It is
believed that genes involved in rapid response
are  more  likely  to  be  co-opted  to  become
toxins than those involved in long term physiological  processes.

The study of platypus venom
shows that there
are other evolutionary processes besides gene
duplication involved in venom evolution, like
mutations,  gene fusions,  alternative  splicing
and domain duplications. Venomous animals
are known to use a diverse array of toxins. To
date, most venom transcriptomic studies have
used expression data to discover novel toxins,
but  next-gen  sequencing  will  likely  become
the  method  of  choice.  Next-gen  sequencing
will advance our understanding of venom adaptation by allowing us to pose specific evolutionary  questions  about  the  evolutionary
mechanisms  responsible  for  venom  adaptation,  the  biological  networks  involved  in
venom production and the adaptability of certain gene families for venom function.
Original publication: Emily S  W Wong  and
Katherine  Belov,  Venom  evolution  through
gene  duplications,
Gene,  Vol.496,  pp.1–7,
2012.

The study of platypus venom
shows that there
are other evolutionary processes besides gene
duplication involved in venom evolution, like
mutations,  gene fusions,  alternative  splicing
and domain duplications. Venomous animals
are known to use a diverse array of toxins. To
date, most venom transcriptomic studies have
used expression data to discover novel toxins,
but  next-gen  sequencing  will  likely  become
the  method  of  choice.  Next-gen  sequencing
will advance our understanding of venom adaptation by allowing us to pose specific evolutionary  questions  about  the  evolutionary
mechanisms  responsible  for  venom  adaptation,  the  biological  networks  involved  in
venom production and the adaptability of certain gene families for venom function.
Original publication: Emily S  W Wong  and
Katherine  Belov,  Venom  evolution  through
gene  duplications,
Gene,  Vol.496,  pp.1–7,
2012.



%%horizontal line
\hrulefill

\sffamily\bfseries\footnotesize Ema Fatima,\\ Centre for Ecological Sciences,\\ 
Indian Institute of  Science,  Bangalore,  India.\\
Email:  fatimaema@gmail.com
\end{multicols}
\end{document}